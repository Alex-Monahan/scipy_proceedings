\section{Introduction}
Labeled image data is essential for training, tuning, and evaluating the performance of many machine learning applications.
Such labels are typically defined with simple polygons, ellipses, and bounding boxes (i.e., ``this rectangle contains a cat").
However, this approach can misrepresent more complex shapes with holes or multiple regions as shown later in \autoref{fig:complexRegion}.
When high accuracy is required, labels must be specified at or close to the pixel-level - a process known as semantic labeling or semantic segmentation.
A detailed description of this process is given in \cite{chengSurveyAnalysisAutomatic2018}.
Examples can readily be found in several popular datasets such as COCO, depicted in \autoref{fig:sampleSegData}.

\makeSampleSegFig

Semantic segmentation is important in numerous domains including printed circuit board assembly (PCBA) inspection (discussed later in the case study) \cite{paradis2020color,azhaganReviewAutomaticBill2019}, quality control during manufacturing \cite{fergusonDetectionSegmentationManufacturing2018,anagnostopoulosComputerVisionApproach2001,anagnostopoulosHighPerformanceComputing2002}, manuscript restoration / digitization \cite{gatosSegmentationfreeRecognitionTechnique2004,kesimanNewSchemeText2016,jainTextSegmentationUsing1992,taxtSegmentationDocumentImages1989,fujisawaSegmentationMethodsCharacter1992}, and effective patient diagnosis \cite{seifertSemanticAnnotationMedical2010,rajchlDeepCutObjectSegmentation2017,yushkevichUserguided3DActive2006,iakovidisRatsnakeVersatileImage2014}.
In all these cases, imprecise annotations severely limit the development of automated solutions and can decrease the accuracy of standard trained segmentation models.

Quality semantic segmentation is difficult due to a reliance on large, high-quality datasets, which are often created by manually labeling each image.
Manual annotation is error-prone, costly, and greatly hinders scalability.
As such, several tools have been proposed to alleviate the burden of collecting these ground-truth labels~\cite{BestImageAnnotation}.
Unfortunately, existing tools are heavily biased toward lower-resolution images with few regions of interest (ROI), similar to \autoref{fig:sampleSegData}.
While this may not be an issue for some datasets, such assumptions are \emph{crippling} for high-fidelity images with hundreds of annotated ROIs~\cite{Ladicky_whatWhereCombiningCRFs,Wang_multiLabelImageAnnotation}.

With improving hardware capabilities and increasing need for high-resolution ground truth segmentation, there are a continually growing number of applications that \emph{require} high-resolution imaging with the previously described characteristics \cite{Mohajerani_cloudRemoteSensing,Demochkina_improvingOneShotXray}.
In these cases, the existing annotation tooling greatly impacts productivity due to the previously referenced assumptions and lack of support \cite{SpaceNet2020-lb}.

In response to these bottlenecks, \emph{we present the Semi-Supervised Semantic Annotation (S3A) annotation and prototyping platform -- an application which eases the process of pixel-level labeling in large, complex scenes.}\footnote{A preliminary version was introduced in an earlier publication~\cite{jessurunComponentDetectionEvaluation2020}, but significant changes to the framework and tool capabilities have been employed since then.}
Its graphical user interface is shown in \autoref{fig:appOverview}.
The software includes live app-level property customization, real-time algorithm modification and feedback, region prediction assistance, constrained component table editing based on allowed data types, various data export formats, and a highly adaptable set of plugin interfaces for domain-specific extensions to S3A.
Beyond software improvements, these features play significant roles in bridging the gap between human annotation efforts and scalable, automated segmentation methods \cite{Branson_humansInLoop}.

\makeAppOverviewFig
